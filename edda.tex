%!TEX TS-program = xelatex
%!TEX encoding = UTF-8 Unicode

\documentclass{scrbook}

\usepackage{xltxtra}
\usepackage{polyglossia}
\usepackage{epigraph}
\usepackage{fancyhdr}
\usepackage{boolexpr}
\usepackage{setspace}
\usepackage{tocloft}
\usepackage{typearea}
\usepackage{perpage}
\usepackage{fontspec}
\usepackage{xunicode}
\usepackage{verbatim} 
\usepackage{color}
\usepackage{microtype}

% ======================================================================
% macro for switching between draft/release modes
\newcounter{draftcounter}
\newcommand{\isDraft}{\boolexpr{\value{draftcounter} = 1}}
\newcommand{\DraftMode}{\setcounter{draftcounter}{1}}

% ======================================================================
% some macro machinery for readability of conditions

\newcounter{bookformatcounter}

\newcommand{\definebookformat}[1]{
	\newcounter{#1}
	\stepcounter{bookformatcounter}
	\setcounter{#1}{\value{bookformatcounter}}
}

\newcommand{\setbookformat}[1]{
	\newcounter{bookformat}
	\setcounter{bookformat}{\value{#1}}
}

\newcommand{\isAFourOneside}{\boolexpr{\value{bookformat} = \value{bfAFourOneside}}}
\newcommand{\isAFourTwoside}{\boolexpr{\value{bookformat} = \value{bfAFourTwoside}}}
\newcommand{\isAFour}{\boolexpr{\isAFourOneside \OR \isAFourTwoside}}

\newcommand{\isIPad}{\boolexpr{\value{bookformat} = \value{bfIPad}}}

\newcommand{\isSevenInches}{\boolexpr{\value{bookformat} = \value{bfSevenInches}}}

% ======================================================================
% select mode here
\DraftMode

% ======================================================================
% select book format here

% available formats
\definebookformat{bfAFourOneside} % A4, one side (for reading on screen)
\definebookformat{bfAFourTwoside} % A4, two sides (for printing)
\definebookformat{bfIPad} % 9.7'' screen, 3x4, 11pt font, portrait
\definebookformat{bfSevenInches} % 7'' screen, 3x4, 11pt font, portrait

\setbookformat{bfAFourOneside}


% ======================================================================
% Size-dependent definitions
% ======================================================================

\switch
\case{\isAFour}
	\ifcase\isAFourOneside
		\usepackage[a4paper, twoside=false]{geometry}
	\else
		\usepackage[a4paper, twoside=true]{geometry}
	\fi
	\KOMAoptions{headings=big, fontsize=11pt}
	\KOMAoptions{headinclude=true, DIV=15}
	\newcommand{\tocnumwidth}{5em}
	\newcommand{\tocspacing}{1.0} % ToC almost fits 2 pages, we need to help it a bit
	\newcommand{\tocmarginwidth}{2.55em} % left indent for non-first lines of ToC
	\newcommand{\titlesize}{48pt}
	\setlength{\parindent}{1.5em}
\case{\isIPad}
	\usepackage[papersize={5.82in,7.76in}, twoside=false]{geometry}
	\areaset{5.5in}{7.44in} % using the whole screen, IPad already has "physical" margin
	\KOMAoptions{headings=normal, fontsize=11pt}
	\KOMAoptions{headinclude=true}
	\newcommand{\tocnumwidth}{5em}
	\newcommand{\tocspacing}{0.9} % ToC almost fits 2 pages, we need to help it a bit
	\newcommand{\tocmarginwidth}{2.55em} % left indent for non-first lines of ToC
	\setlength{\parindent}{1em}
\case{\isSevenInches}
	\usepackage[papersize={4.8in,5.6in}, twoside=false]{geometry}
	\areaset{4.5in}{5.4in} % using the whole screen, readers have "physical" margin
	\KOMAoptions{headings=normal, fontsize=11pt}
	\KOMAoptions{headinclude=true}
	\newcommand{\tocnumwidth}{5em}
	\newcommand{\tocspacing}{0.9}
	\newcommand{\tocmarginwidth}{2.55em} % left indent for non-first lines of ToC
	\setlength{\parindent}{1em}
\endswitch
\recalctypearea


% ======================================================================
% Size-independent definitions
% ======================================================================

\setmainfont[Mapping=tex-text]{Charis SIL}
\setsansfont[Mapping=tex-text]{Optima}

\newfontfamily\numberfont[Mapping=tex-text]{Optima}
\newfontfamily\commentfont[Mapping=tex-text]{Charis SIL}
\newfontfamily\stanzafont[Mapping=tex-text]{Gentium}

\setmainlanguage{english}
\setotherlanguage{icelandic}
\pagestyle{fancy} % enable fancy headers

% FIXME: probably more accurate tuning is needed
% but this values do not produce overfull hboxes, so they are good enough for a start
% (otherwise I would have to wrap single paragraphs in \sloppy...\fussy)

% tolerance for paragraph badness
\tolerance = 1000

% if two first passes fail and this value is nonzero, TeX will perform the third pass, using additional whitespace
\emergencystretch = 20pt	

% prefer uneven page borders and constant distance between paragraphs
\raggedbottom

\MakePerPage{footnote} % reset footnote counter on each page

% try not to allow cases when last paragraph line starts a new page or when first paragraph line ends a page
\widowpenalty=1300
\clubpenalty=1300

% proper hyphenation for complex words
\XeTeXinterchartokenstate=1
\XeTeXcharclass `\- 24
\XeTeXinterchartoks 24 0 = {\hskip\z@skip}
\XeTeXinterchartoks 0 24 = {\nobreak}

\newcommand{\mdash}{---}
\newcommand{\ndash}{--}
\newcommand{\sdash}{---} % dash for the beginning of speech (babel uses "--* for it)
\newcommand{\enummdash}{---} % mdash for enumerations
\newcommand{\commamdash}{---} % mdash after comma

% ======================================================================
% Stanza magic
% ======================================================================

% in case you want to change the way stanza number is displayed
\newcommand{\displaystanzanum}[1]{{\numberfont\Large\textbf{#1.}}}

\newcommand{\mystanzapair}[4][\@empty]{

\phantomsection
\label{\thischapterlabel:#2}

\hskip -3\parindent
\begin{tabular}{@{} r @{} p{0.48\textwidth} @{} p{0.5\textwidth}}

\makebox[2\parindent][r]{\displaystanzanum{#2}\quad}

&

\begin{minipage}[t]{0.46\textwidth}
\setstretch{1.3}\selectfont\texticelandic{\stanzafont#3}
\end{minipage}

&

\begin{minipage}[t]{0.5\textwidth}
\setstretch{1.3}\selectfont\textenglish{\stanzafont#4}
\end{minipage} 

\end{tabular}

\ifx\@empty#1
\else
\vskip \baselineskip
\commentfont#1
\fi
\vskip 2\baselineskip

}

\newcommand{\myverse}[1]{
	{\stanzafont
		\begin{verse}
		#1
		\end{verse}
	}
}

\newcommand{\thischapterlabel}{Stub value, should not see it}
\newcommand{\eddachapter}[3]{
	\addchap[#1 (#3)]{#1}
	\vskip -\baselineskip
	{\Large\sffamily #3}
	\vskip 2\baselineskip
	\renewcommand{\thischapterlabel}{cha:#2}
	\label{\thischapterlabel}

	% Additional label variants
	% Rationale:
	% 1. I want labels to appear in text in italic.
	% 2. Punctuation near italic text must be italic too.
	% 3. \hyperref does not support complex macros (I tired to use xstring to cut punctuation from labels) as labels (probably one has to somehow explicitly evaluate them, but I did not manage to do that).
	% 4. I want to hide all complexity in macros (and writing something like \chapterref{Voluspo}{,} is just ugly).
	% Therefore it seems like the simplest solution.
	\label{\thischapterlabel.}
	\label{\thischapterlabel,}
	\label{\thischapterlabel:}
	\label{\thischapterlabel;}
}

% No spaces in these macros, because TeX considers them to be significant

\ifcase\isDraft
	\newcommand{\chapterref}[1]{\textcolor{red}{\emph{#1}}}
	\newcommand{\stanzaref}[2][\@empty]{\textcolor{red}{#2}}
\else
	\newcommand{\chapterref}[1]{\hyperref[cha:#1]{\emph{#1}}}
	\newcommand{\stanzaref}[2][\@empty]{\ifx\@empty#1\hyperref[\thischapterlabel:#2]{#2}\else\hyperref[cha:#1:#2]{#2}\fi}
\fi

\newcommand{\stanzarangeref}[3][\@empty]{\stanzaref[#1]{#2}\ndash\stanzaref[#1]{#3}}
\newcommand{\combinedref}[2]{\chapterref{#1,} \stanzaref[#1]{#2}}
\newcommand{\combinedrangeref}[3]{\chapterref{#1,} \stanzarangeref[#1]{#2}{#3}}
	
\newcommand{\lacuna}{. . .}

% separator for half-lines in stanzas
\definecolor{gray}{cmyk}{0,0,0,0.5}
\newcommand{\sep}{\textcolor{gray}{~|~}}

\newcommand*\sepline{
	\begin{center}
	\rule[1ex]{.5\textwidth}{.5pt}
	\end{center}
}

% Pdf converter settings
\usepackage{cmap}
\usepackage[bookmarks, bookmarkstype=toc, pdfborder={0 0 0}, pdfa=true]{hyperref}
\hypersetup{pdftitle={Poetic Edda}}

% ======================================================================
% Main document 
% ======================================================================
\begin{document}

% title
\title{\fontsize{\titlesize}{\titlesize}\selectfont Poetic Edda}
\subtitle{\Large{Icelandic{\ndash}English diglot}}
\author{
Original text from Konungsbók, \\
Hauksbók and Codex Regius; \\
translation and comments by \\
Henry Adams Bellows
}

\date{}
\publishers{\small{Typeset by Bogdan Opanchuk, \\ compiled on \today \\ Melbourne, Australia}}

\maketitle

\frontmatter

\clearpage
\begingroup

	% Disable page numbers
	\pagestyle{empty}
	\renewcommand*{\chapterpagestyle}{empty}
	\tocloftpagestyle{empty} % due to tocloft package, ToC is not affected by common page style

	\renewcommand{\@pnumwidth}{\tocnumwidth}
	\renewcommand{\@tocrmarg}{\tocmarginwidth plus1fil} % make ToC flushed to the left

	\begin{spacing}{\tocspacing}
	\tableofcontents
	\end{spacing}

\clearpage
\endgroup

\mainmatter

\addchap{General Introduction}

There is scarcely any literary work of great importance which has been less readily available for the general reader, or even for the serious student of literature, than the Poetic Edda. Translations have been far from numerous, and only in Germany has the complete work of translation been done in the full light of recent scholarship. In English the only versions were long the conspicuously inadequate one made by Thorpe, and published about half a century ago, and the unsatisfactory prose translations in Vigfusson and Powell's \emph{Corpus Poeticum Boreale,} reprinted in the Norrœna collection. An excellent translation of the poems dealing with the gods, in verse and with critical and explanatory notes, made by Olive Bray, was, however, published by the Viking Club of London in 1908. In French there exist only partial translations, chief among them being those made by Bergmann many years ago. Among the seven or eight German versions, those by the Brothers Grimm and by Karl Simrock, which had considerable historical importance because of their influence on nineteenth century German literature and art, and particularly on the work of Richard Wagner, have been largely superseded by Hugo Gering's admirable translation, published in 1892, and by the recent two volume rendering by Genzmer, with excellent notes by Andreas Heusler, 1834\ndash1921. There are competent translations in both Norwegian and Swedish. The lack of any complete and adequately annotated English rendering in metrical form, based on a critical text, and profiting by the cumulative labors of such scholars as Mogk, Vigfusson, Finnur Jonsson, Grundtvig, Bugge, Gislason, Hildebrand, Lüning, Sweet, Niedner, Ettmüller, Müllenhoff, Edzardi, B. M. Olsen, Sievers, Sijmons, Detter, Heinzel, Falk, Neckel, Heusler, and Gering, has kept this extraordinary work practically out of the reach of those who have had neither time nor inclination to master the intricacies of the original Old Norse.

On the importance of the material contained in the \emph{Poetic Edda} it is here needless to dwell at any length. We have inherited the Germanic traditions in our very speech, and the \emph{Poetic Edda} is the original storehouse of Germanic mythology. It is, indeed, in many ways the greatest literary monument preserved to us out of the antiquity of the kindred races which we call Germanic. Moreover, it has a literary value altogether apart from its historical significance. The mythological poems include, in the \chapterref{Voluspo,} one of the vastest conceptions of the creation and ultimate destruction of the world ever crystallized in literary form; in parts of the \chapterref{Hovamol,} a collection of wise counsels that can bear comparison with most of the Biblical Book of Proverbs; in the \chapterref{Lokasenna,} a comedy none the less full of vivid characterization because its humor is often broad; and in the \chapterref{Thrymskvitha,} one of the finest ballads in the world. The hero poems give us, in its oldest and most vivid extant form, the story of Sigurth, Brynhild, and Atli, the Norse parallel to the German \emph{Nibelungenlied.} The Poetic Edda is not only of great interest to the student of antiquity; it is a collection including some of the most remark able poems which have been preserved to us from the period before the pen and the printing-press. replaced the poet-singer and oral tradition. It is above all else the desire to make better known the dramatic force, the vivid and often tremendous imagery, and the superb conceptions embodied in these poems which has called forth the present translation.

\addsec*{What is the Poetic Edda?}

Even if the poems of the so-called Edda were not so significant and intrinsically so valuable, the long series of scholarly struggles which have been going on over them for the better part of three centuries would in itself give them a peculiar interest. Their history is strangely mysterious. We do not know who composed them, or when or where they were composed; we are by no means sure who collected them or when he did so; finally, we are not absolutely certain as to what an ``Edda'' is, and the best guess at the meaning of the word renders its application to this collection of poems more or less misleading.

A brief review of the chief facts in the history of the \emph{Poetic Edda} will explain why this uncertainty has persisted. Preserved in various manuscripts of the thirteenth and early fourteenth centuries is a prose work consisting of a very extensive collection of mythological stories, an explanation of the important figures and tropes of Norse poetic diction,{\commamdash}the poetry of the Icelandic and Norwegian skalds was appallingly complex in this respect,{\commamdash}and a treatise on metrics. This work, clearly a handbook for poets, was commonly known as the ``Edda'' of Snorri Sturluson, for at the head of the copy of it in the \emph{Uppsalabok,} a manuscript written presumably some fifty or sixty years after Snorri's death, which was in 1241, we find: ``This book is called Edda, which Snorri Sturluson composed.'' This work, well known as the \emph{Prose Edda,} Snorri's \emph{Edda} or the \emph{Younger Edda,} has recently been made available to readers of English in the admirable translation by Arthur G. Brodeur, published by the American-Scandinavian Foundation in 1916.

Icelandic tradition, however, persisted in ascribing either this \emph{Edda} or one resembling it to Snorri's much earlier compatriot, Sæmund the Wise (1056\ndash1133). When, early in the seventeenth century, the learned Arngrimur Jonsson proved to everyone's satisfaction that Snorri and nobody else must have been responsible for the work in question, the next thing to determine was what, if anything, Sæmund had done of the same kind. The nature of Snorri's book gave a clue. In the mythological stories related a number of poems were quoted, and as these and other poems were to all appearances Snorri's chief sources of information, it was assumed that Sæmund must have written or compiled a verse \emph{Edda}{\mdash}whatever an ``Edda'' might be{\mdash}on which Snorri's work was largely based.

So matters stood when, in 1643, Brynjolfur Sveinsson, Bishop of Skalholt, discovered a manuscript, clearly written as early as 1300, containing twenty-nine poems, complete or fragmentary, and some of them with the very lines and stanzas used by Snorri. Great was the joy of the scholars, for here, of course, must be at least a part of the long-sought \emph{Edda} of Sæmund the Wise. Thus the good bishop promptly labeled his find, and as Sæmund's \emph{Edda,} the \emph{Elder Edda} or the \emph{Poetic Edda} it has been known to this day.

This precious manuscript, now in the Royal Library in Copenhagen, and known as the \emph{Codex Regius} (R2365), has been the basis for all published editions of the Eddic poems. A few poems of similar character found elsewhere have subsequently been added to the collection, until now most editions include, as in this translation, a total of thirty-four. A shorter manuscript now in the Arnamagnæan collection in Copenhagen (AM748), contains fragmentary or complete versions of six of the poems in the \emph{Codex Regius,} and one other, \emph{Baldrs Draumar,} not found in that collection. Four other poems (\chapterref{Rigsthula,} \chapterref{Hyndluljoth,} \emph{Grougaldr} and \emph{Fjolsvinnsmol,} the last two here combined under the title of \chapterref{Svipdagsmol}), from various manuscripts, so closely resemble in subject-matter and style the poems in the \emph{Codex Regius} that they have been included by most editors in the collection. Finally, Snorri's \emph{Edda} contains one complete poem, the \chapterref{Grottasongr,} which many editors have added to the poetic collection; it is, however, not included in this translation, as an admirable English version of it is available in Mr. Brodeur's rendering of Snorri's work.

From all this it is evident that the \emph{Poetic Edda,} as we now know it, is no definite and plainly limited work, but rather a more or less haphazard collection of separate poems, dealing either with Norse mythology or with hero-cycles unrelated to the traditional history of greater Scandinavia or Iceland. How many other similar poems, now lost, may have existed in such collections as were current in Iceland in the later twelfth and thirteenth centuries we cannot know, though it is evident that some poems of this type are missing. We can say only that thirty-four poems have been preserved, twenty-nine of them in a single manuscript collection, which differ considerably in subject-matter and style from all the rest of extant Old Norse poetry, and these we group together as the \emph{Poetic Edda.}

But what does the word ``Edda'' mean? Various guesses have been made. An early assumption was that the word somehow meant ``Poetics,'' which fitted Snorri's treatise to a nicety, but which, in addition to the lack of philological evidence to support this interpretation, could by no stretch of scholarly subtlety be made appropriate to the collection of poems. Jacob Grimm ingeniously identified the word with the word ``edda'' used in one of the poems, the \chapterref{Rigsthula,} where, rather conjecturally, it means ``great-grand mother.'' The word exists in this sense no where else in Norse literature, and Grimm's suggestion of ``Tales of a Grandmother,'' though at one time it found wide acceptance, was grotesquely. inappropriate to either the prose or the verse work.

At last Eirikr Magnusson hit on what appears the likeliest solution of the puzzle: that ``Edda'' is simply the genitive form of the proper name ``Oddi.'' Oddi was a settlement in the southwest of Iceland, certainly the home of Snorri Sturluson for many years, and, traditionally at least, also the home of Sæmund the Wise. That Snorri's work should have been called ``The Book of Oddi'' is altogether reasonable, for such a method of naming books was common{\mdash}witness the ``Book of the Flat Island'' and other early manuscripts. That Sæmund may also have written or compiled another ``Oddi-Book'' is perfectly possible, and that tradition should have said he did so is entirely natural.

It is, however, an open question whether or not Sæmund had anything to do with making the collection, or any part of it, now known as the Poetic Edda, for of course the seventeenth-century assignment of the work to him is negligible. p. xvii We can say only that he may have made some such compilation, for he was a diligent student of Icelandic tradition and history, and was famed throughout the North for his learning. But otherwise no trace of his works survives, and as he was educated in Paris, it is probable that he wrote rather in Latin than in the vernacular.

All that is reasonably certain is that by the middle or last of the twelfth century there existed in Iceland one or more written collections of Old Norse mythological and heroic poems, that the \emph{Codex Regius,} a copy made a hundred years or so later, represents at least a considerable part of one of these, and that the collection of thirty-four poems which we now know as the \emph{Poetic} or \emph{Elder Edda} is practically all that has come down to us of Old Norse poetry of this type. Anything more is largely guesswork, and both the name of the compiler and the meaning of the title ``Edda'' are conjectural.

\addsec*{The origin of the Eddic poems}

There is even less agreement about the birthplace, authorship and date of the Eddic poems themselves than about the nature of the existing collection. Clearly the poems were the work of many different men, living in different periods; clearly, too, most of them existed in oral tradition for generations before they were committed to writing. In general, the mythological poems seem strongly marked by pagan sincerity, although efforts have been made to prove them the results of deliberate archaizing; and as Christianity became generally accepted throughout the Norse world early in the eleventh century, it seems altogether likely that most of the poems dealing with the gods definitely antedate the year 1000. The earlier terminus is still a matter of dispute. The general weight of critical opinion, based chiefly on the linguistic evidence presented by Hoffory, Finnur Jonsson and others, has indicated that the poems did not assume anything closely analogous to their present forms prior to the ninth century. On the other hand, Magnus Olsen's interpretation of the inscriptions on the Eggjum Stone, which he places as early as the seventh century, have led so competent a scholar as Birger Nerman to say that ``we may be warranted in concluding that some of the Eddic poems may have originated, wholly or partially, in the second part of the seventh century.'' As for the poems belonging to the hero cycles, one or two of them appear to be as late as 1100, but most of them probably date back at least to the century and a half following 900. It is a reasonable guess that the years between 850 and 1050 saw the majority of the Eddic poems worked into definite shape, but it must be remembered that many changes took place during the long subsequent period of oral transmission, and also that many of the legends, both mythological and heroic, on which the poems were based certainly existed in the Norse regions, and quite possibly in verse form, long before the year 900.

As to the origin of the legends on which the poems are based, the whole question, at least so far as the stories of the gods are concerned, is much too complex for discussion here. How much of the actual narrative material of the mythological lays is properly to be called Scandinavian is a matter for students of comparative mythology to guess at. The tales underlying the heroic lays are clearly of foreign origin: the Helgi story comes from Denmark, and that of Völund from Germany, as also the great mass of traditions centering around Sigurth (Siegfried), Brynhild, the sons of Gjuki, Atli (Attila), and Jormunrek (Ermanarich). The introductory notes to the various poems deal with the more important of these questions of origin. Of the men who composed these poems{\commamdash}``wrote'' is obviously the wrong word{\mdash}we know absolutely nothing, save that some of them must have been literary artists with a high degree of conscious skill. The Eddic poems are ``folk-poetry,''{\mdash}whatever that may be,{\commamdash}only in the sense that some of them strongly reflect racial feelings and beliefs; they are anything but crude or primitive in workmanship, and they show that not only the poets themselves, but also many of their hearers, must have made a careful study of the art of poetry.

Where the poems were shaped is equally uncertain. Any date prior to 875 would normally imply an origin on the mainland, but the necessarily fluid state of oral tradition made it possible for a poem to be ``composed'' many times over, and in various and far-separated places, without altogether losing its identity. Thus, even if a poem first assumed something approximating its present form in Iceland in the tenth century, it may none the less embody language characteristic of Norway two centuries earlier. Oral poetry has always had an amazing preservative power over language, and in considering the origins of such poems as these, we must cease thinking in terms of the printing-press, or even in those of the scribe. The claims of Norway as the birthplace of most of the Eddic poems have been extensively advanced, but the great literary activity of Iceland after the settlement of the island by Norwegian emigrants late in the ninth century makes the theory of an Icelandic home for many of the poems appear plausible. The two Atli lays, with what authority we do not know, bear in the \emph{Codex Regius} the superscription ``the Greenland poem,'' and internal evidence suggests that this statement may be correct. Certainly in one poem, the \chapterref{Rigsthula,} and probably in several others, there are marks of Celtic influence. During a considerable part of the ninth and tenth centuries, Scandinavians were active in Ireland and in most of the western islands inhabited by branches of the Celtic race. Some scholars have, indeed, claimed nearly all the Eddic poems for these ``Western Isles.'' However, as Iceland early came to be the true cultural center of this Scandinavian island world, it may be said that the preponderant evidence concerning the development of the Eddic poems in anything like their present form points in that direction, and certainly it was in Iceland that they were chiefly preserved.

\addsec*{The Edda and Old Norse literature}

Within the proper limits of an introduction it would be impossible to give any adequate summary of the history and literature with which the Eddic poems are indissolubly connected, but a mere mention of a few of the salient facts may be of some service to those who are unfamiliar with the subject. Old Norse literature covers approximately the period between 850 and 1300. During the first part of that period occurred the great wanderings of the Scandinavian peoples, and particularly the Norwegians. A convenient date to remember is that of the sea-fight of Hafrsfjord, 872, when Harald the Fair-Haired broke the power of the independent Norwegian nobles, and made himself overlord of nearly all the country. Many of the defeated nobles fled overseas, where inviting refuges had been found for them by earlier wanderers and plunder-seeking raiders. This was the time of the inroads of the dreaded Northmen in France, and in 885 Hrolf Gangr (Rollo) laid siege to Paris itself. Many Norwegians went to Ireland, where their compatriots had already built Dublin, and where they remained in control of most of the island till Brian Boru shattered their power at the battle of Clontarf in 1014.

Of all the migrations, however, the most important were those to Iceland. Here grew up an active civilization, fostered by absolute independence and by remoteness from the wars which wracked Norway, yet kept from degenerating into provincialism by the roving life of the people, which brought them constantly in contact with the culture of the South. Christianity, introduced throughout the Norse world about the year 1000, brought with it the stability of learning, and the Icelanders became not only the makers but also the students and recorders of history. The years between 875 and 1100 were the great spontaneous period of oral literature. Most of the military and political leaders were also poets, and they composed a mass of lyric poetry concerning the authorship of which we know a good deal, and much of which has been preserved. Narrative prose also flourished, for the Icelander had a passion for story-telling and story-hearing. After 1100 came the day of the writers. These sagamen collected the material that for generations had passed from mouth to mouth, and gave it permanent form in writing. The greatest bulk of what we now have of Old Norse literature{\commamdash}and the published part of it makes a formidable library{\commamdash}originated thus in the earlier period before the introduction of writing, and was put into final shape by the scholars, most of them Icelanders, of the hundred years following 1150.

After 1250 came a rapid and tragic decline. Iceland lost its independence, becoming a Norwegian province. Later Norway too fell under alien rule, a Swede ascending the Norwegian throne in 1320. Pestilence and famine laid waste the whole North; volcanic disturbances worked havoc in Iceland. Literature did not quite die, but it fell upon evil days; for the vigorous native narratives and heroic poems of the older period were substituted translations of French romances. The poets wrote mostly doggerel; the prose writers were devoid of national or racial inspiration.

The mass of literature thus collected and written down largely between 1150 and 1250 maybe roughly divided into four groups. The greatest in volume is made up of the sagas: narratives mainly in prose, ranging all the way from authentic history of the Norwegian kings and the early Icelandic settlements to fairy-tales. Embodied in the sagas is found the material composing the second group: the skaldic poetry, a vast collection of songs of praise, triumph, love, lamentation, and so on, almost uniformly characterized by an appalling complexity of figurative language. There is no absolute line to be drawn between the poetry of the skalds and the poems of the \emph{Edda,} which we may call the third group; but in addition to the remarkable artificiality of style which marks the skaldic poetry, and which is seldom found in the poems of the \emph{Edda,} the skalds dealt almost exclusively with their own emotions, whereas the Eddic poems are quite impersonal. Finally, there is the fourth group, made up of didactic works, religious and legal treatises, and so on, studies which originated chiefly in the later period of learned activity.

\addsec*{Preservation of the Eddic poems}

Most of the poems of the \emph{Poetic Edda} have unquestionably reached us in rather bad shape. During the long period of oral transmission they suffered all sorts of interpolations, omissions and changes, and some of them, as they now stand, are a bewildering hodge-podge of little related fragments. To some extent the diligent twelfth century compiler to whom we owe the \emph{Codex Regius}{\ndash}Sæmund or another{\mdash}was himself doubtless responsible for the patchwork process, often supplemented by narrative prose notes of his own; but in the days before written records existed, it was easy to lose stanzas and longer passages from their context, and equally easy to interpolate them where they did not by any means belong. Some few of the poems, however, appear to be virtually complete and unified as we now have them.

Under such circumstances it is clear that the establishment of a satisfactory text is a matter of the utmost difficulty. As the basis for this translation I have used the text prepared by Karl Hildebrand (1876) and revised by Hugo Gering (1904). Textual emendation has, however, been so extensive in every edition of the \emph{Edda,} and has depended so much on the theories of the editor, that I have also made extensive use of many other editions, notably those by Finnur Jonsson, Neckel, Sijmons, and Detter and Heinzel, together with numerous commentaries. The condition of the text in both the principal codices is such that no great reliance can be placed on the accuracy of the copyists, and frequently two editions will differ fundamentally as to their readings of a given passage or even of an entire-poem. For this reason, and because guesswork necessarily plays so large a part in any edition or translation of the Eddic poems, I have risked overloading the pages with textual notes in order to show, as nearly as possible, the exact state of the original together with all the more significant emendations. I have done this particularly in the case of transpositions, many of which appear absolutely necessary, and in the indication of passages which appear to be interpolations.

\addsec*{The verse-forms of the Eddic poems}

The many problems connected with the verse-forms found in the Eddic poems have been analyzed in great detail by Sievers, Neckel, and others. The three verse-forms exemplified in the poems need only a brief comment here, however, in order to make clear the method used in this translation. All of these forms group the lines normally in four-line stanzas. In the so-called Fornyrthislag (``Old Verse''), for convenience sometimes referred to in the notes as four-four measure, these lines have all the same structure, each line being sharply divided by a cæsural pause into two half-lines, and each half-line having two accented syllables and two (sometimes three) unaccented ones. The two half-lines forming a complete line are bound together by the alliteration, or more properly initial-rhyme, of three (or two) of the accented syllables. The following is an example of the Fornyrthislag stanza, the accented syllables being in italics:

\myverse{
\textit{Vreiþr} vas \textit{Ving}þórr,{\sep}es \textit{vakna}þi \\
ok \textit{síns} \textit{ham}ars{\sep}of \textit{sakna}þi; \\
\textit{skegg} nam \textit{hris}ta,{\sep}\textit{skǫr} nam \textit{dý}ja, \\
\textit{ré}þ \textit{Jarþ}ar \textit{burr}{\sep}\textit{umb} at \textit{þreif}ask.
}

In the second form, the Ljothahattr (``Song Measure''), the first and third line of each stanza are as just described, but the second and fourth are shorter, have no cæsural pause, have three accented syllables, and regularly two initial-rhymed accented syllables, for which reason I have occasionally referred to Ljothahattr as four-three measure. The following is an example:

\myverse{
\textit{Ar} skal \textit{rí}sa{\sep}sás \textit{an}nars \textit{vill} \\
\hskip 1.5em \textit{fé} eþa \textit{fiǫr} ha\textit{fa;} \\
\textit{ligg}jandi \textit{ulfr}{\sep}sjaldan \textit{láer} of \textit{getr} \\
\hskip 1.5em né \textit{sof}andi \textit{maþr} \textit{sigr.}
}

In the third and least commonly used form, the Malahattr (``Speech Measure''), a younger verse-form than either of the other two, each line of the four-line stanza is divided into two half-lines by a cæsural pause, each half line having two accented syllables and three (sometimes four) unaccented ones; the initial rhyme is as in the Fornyrthislag. The following is an example:

\myverse{
\textit{Horsk} vas \textit{hús}freyja,{\sep}\textit{hug}þi at \textit{mann}viti, \\
\textit{lag} heyrþi \textit{òr}þa,{\sep}hvat á \textit{laun} \textit{máel}tu; \\
þá vas \textit{vant} \textit{vit}ri,{\sep}\textit{vil}di þeim \textit{hjal}þa: \\
skyldu of \textit{sáe} \textit{sig}la,{\sep}en \textit{sjǫlf} né \textit{kvamsk}at.
}

A poem in Fornyrthislag is normally entitled \emph{-kvitha} (\chapterref{Thrymskvitha,} \chapterref{Guthrunarkvitha,} etc.), which for convenience I have rendered as ``lay,'' while a poem in Ljothahattr is entitled \emph{-mol} (\chapterref{Grimnismol,} \chapterref{Skirnismol,} etc.), which I have rendered as ``ballad.'' It is difficult to find any distinction other than metrical between the two terms, although it is clear that one originally existed.

Variations frequently appear in all three kinds of verse, and these I have attempted to indicate through the rhythm of the translation. In order to preserve so far as possible the effect of the Eddic verse, I have adhered, in making the English version, to certain of the fundamental rules governing the Norse line and stanza formations. The number of lines to each stanza conforms to what seems the best guess as to the original, and I have consistently retained the number of accented syllables. in translating from a highly inflected language into one depending largely on the use of subsidiary words, it has, however, been necessary to employ considerable freedom as to the number of unaccented syllables in a line. The initial-rhyme is generally confined to two accented syllables in each line. As in the original, all initial vowels are allowed to rhyme interchangeably, but I have disregarded the rule which lets certain groups of consonants rhyme only with themselves \emph{(e.g.,} I have allowed initial \emph{s} or \emph{st} to rhyme with \emph{sk} or \emph{sl)}. In general, I have sought to preserve the effect of the original form whenever possible without an undue sacrifice of accuracy. For purposes of comparison, the translations of the three stanzas just given are here included:

Fornyrthislag:

\myverse{
\textit{Wild} was \textit{Ving}thor{\sep}\textit{when} he \textit{awoke,} \\
And \textit{when} his \textit{might}y{\sep}\textit{ham}mer he \textit{missed;} \\
He \textit{shook} his \textit{beard,}{\sep}his \textit{hair} was \textit{brist}ling, \\
To \textit{grop}ing \textit{set}{\sep}the \textit{son} of \textit{Jorth.}
}

Ljothahattr:

\myverse{
He must \textit{ear}ly go \textit{forth}{\sep}who \textit{fain} the \textit{blood} \\
Or the \textit{goods} of an\textit{other} would \textit{get;} \\
The \textit{wolf} that lies \textit{id}le{\sep}shall \textit{win} little \textit{meat,} \\
Or the \textit{sleep}ing man suc\textit{cess.}
}

Malahattr:

\myverse{
\textit{Wise} was the \textit{wom}an,{\sep}she \textit{fain} would use \textit{wis}dom, \\
She \textit{saw} well what \textit{meant}{\sep}all they \textit{said} in \textit{sec}ret; . . \\
From her \textit{heart} it was \textit{hid}{\sep}how \textit{help} she might \textit{ren}der, \\
The \textit{sea} they should \textit{sail,}{\sep}while her\textit{self} she should \textit{go} not.
}

\addsec*{Proper Names}

The forms in which the proper names appear in this translation will undoubtedly perplex and annoy those who have become accustomed to one or another of the current methods of anglicising Old Norse names. The nominative ending -r it has seemed best to, omit after consonants, although it has been retained after vowels; in Baldr the final -r is a part of the stem and is of course retained. I have rendered the Norse Þ by ``th'' throughout, instead of spasmodically by ``d,'' as in many texts: \emph{e.g.,} Othin in stead of Odin. For the Norse ø I have used its equivalent, ``ö,'' \emph{e.g.,} Völund; for the o I have used ``o'' and not ``a,'' e.g., Voluspo, not Valuspa or Voluspa. To avoid confusion with accents the long vowel marks of the Icelandic are consistently omitted, as likewise in modern Icelandic proper names. The index at the end of the book indicates the pronunciation in each case.

\addsec*{Conclusion}

That this translation may be of some value to those who can read the poems of the \emph{Edda} in the original language I earnestly hope. Still more do I wish that it may lead a few who hitherto have given little thought to the Old Norse language and literature to master the tongue for themselves. But far above either of these I place the hope that this English version may give to some, who have known little of the ancient traditions of what is after all their own race, a clearer insight into the glories of that extraordinary past, and that I may through this medium be able to bring to others a small part of the delight which I myself have found in the poems of the \emph{Poetic Edda.}

% ======================================================================
% Chapters
% ======================================================================

\include{voluspo}
%\include{hovamol}

\end{document}